\documentclass{article}
\usepackage{geometry}
\usepackage{graphicx}
\geometry{left=3cm, right=3cm, top=2cm, bottom=2cm}
\title{Write-up}
\author{Yu Gu}
\date{\today}
\setlength{\parindent}{0pt}\begin{document}
\maketitle

The basic problem of this project is to analyze the stability of a 2D truss and compute the beam forces. I design a python class Truss to deal with this task. Besides  \emph{\_\_init\_\_()} method to call the function of the whole process and \emph{\_\_repr\_\_()} method to print out the result, this Truss class contains 5 methods as the decomposition of this task.

\setlength{\parindent}{0pt}\begin{enumerate}
\item \emph{solve(joints\_file, beams\_file, output\_file)}:\\
This method is to implement all steps to do truss analysis. The input arguments are the directory and filename of data and output figure. The output\_file is optional and default = False.

\item \emph{load\_file(joints\_file, beams\_file)}:\\
This method is to load the data of joints and beams from input file. The input arguments are the directory and filename of joints and beams data.
\begin{itemize}
\item Use \emph{np.loadtxt(filename, skiprows=1)} to load the data and skip the first row which is the name of the variables.
\item Generate a dictionary \emph{xy} by extracting the coordinate of joints.
\item Store number of joints and beams in the class.
\end{itemize}

\item \emph{PlotGeometry(output\_file)}:\\
This method is to plot the geometry of the truss. The input argument is the directory and filename of output figure. The idea is: while output\_file is not False. For each beam, extract the coordinates of two joints, then plot beams as blue line and plot joints as red point. 

\item \emph{sparse\_matrix()}:\\
This method is to create sparse matrix of the linear system of equations. The idea of designing the linear system Ax=b is: \\\\
We have 2 types of equations: (1) equilibrium of each joint in 2 directions; (2) geometric constraints for beam force of each beam. Hence, there should be (2*n\_joints+n\_beams) equations. \\\\
For each equation, the variables are 2 types of unknown forces: (1) beams forces of each beam in 2 directions; (2) reaction forces of each fixed joint in 2 directions. Then, there should be (2*n\_beams+2*n\_support) variables. \\\\
As for the known external force, we put it as b. Thus, the matrix A are of size(2*n\_joints+n\_beams, 2*n\_beams+2*n\_support). Comparing with setting directional beam forces in x and putting geometric constrains as coefficients into equilibrium equation, the matrix in our system has larger size, but the entries of A in the first two columns are just 1 or -1, this is much easier to generate the matrix.
\begin{itemize}
\item  The first 2*n\_joints rows are the equilibrium equations:\\
Row i and i+n\_joints are equations for the ith joint in x and y directions.
\item The last n\_beams rows are the geometric constraint equations:\\
Row i is equation for the (i-2*n\_joints)th beam.
\item The first 2*n\_beams columns are coefficients of beam forces:\\
Column j and j+n\_beams are coefficients of jth beam in x and y directions.
\item The last 2*n\_support are coefficients of reaction forces:\\
Column j and j+n\_support are coefficients of (j-2*n\_beams)th fixed joint in x and y directions.
\end{itemize}
Because each joint only belongs to a few beams, and each geometric constraint also stands for only one beam. So there are many 0 appears in A. Thus, we try to use sparse matrix to describe A. First, we generate the sparse matrix in COO  format.
\begin{itemize}
\item Get entries for the submatrix A[0:2*n\_joints, 0:2*n\_beams]\\
For each beam, the Bx, By coefficient of the first joint is 1, and -1 for the second joint.\\
For each joint, related beams can be divided into 2 types by the position of the joint.
\item Get entries for the submartix [2*n\_joints:2*n\_joints+b\_beams, 0:2*n\_beams]\\
For each beam, the Bx, By coefficient of the first joint is -dy, and dx for the second joint.
\item Get entries for the submatrix [0:2*n\_joints, 2*n\_beams:2*n\_beams+2*n\_support]\\
For each fixed joint, the Rx, Ry coeffcient is 1.
\item Finally, convert sparse matrix to CSR format for later computing.
\end{itemize}

\item \emph{compute\_force()}:
This method is to compute the beam force by solving the linear system of equations. We will raise error messages in 2 cases:
\begin{itemize}
\item The matrix is not square, which means this truss geometry not suitable for static equilibrium analysis.
\item The matrix is singular, which means the truss  is unstable.
\end{itemize}
After solving x by \emph{sparse.linalg.spslove(A, b)}, we need to compute the directional beam forces. The sign of the beam forces are determined by the sign of dot product between (dx, dy) and (Bx, By).
\end{enumerate}

By doing all these 5 methods above, we can get the result of truss analysis: whether the truss is stable, and the equilibrium beam forces while stable. Figure below shows the geometric of sample truss 2.
\begin{figure}[h]
\centering
\includegraphics[width=12cm, height=9cm]{result.png}
\caption{Geometry of Truss 2}
\end{figure}

\end{document}
