\documentclass{article}
\usepackage{geometry}
\usepackage{graphicx}
\usepackage{algorithm}
\usepackage{algorithmic}
\geometry{left=3cm, right=3cm, top=2cm, bottom=2cm}
\title{Write-up}
\author{Yu Gu}
\date{\today}
\setlength{\parindent}{0pt}\begin{document}
\maketitle

\begin{enumerate}
\item \textbf{Pseudo-code for Conjugate Gradient (CG) algorithm}
\begin{algorithm}
\caption{CG algorithm} 
\begin{algorithmic}
    \STATE \textbf{Initialize}: $u_0$
    \STATE $r_0=b-Au_0$
    \STATE $p_0=r_0$
    \STATE $niter=0$
    \WHILE{$niter<nitermax$}
        \STATE $niter=niter + 1$
        \STATE $\alpha_n={r_n}^Tr_n/{p_n}^TAp_n$
        \STATE $u_{n+1}=u_n+\alpha_np_n$
        \STATE $r_{n+1}=r_n-\alpha_nAp_n$
        \IF{$\left\|r_{n+1}\right\|_2/\left\|r_0\right\|_2<threshold$}
            \STATE \textbf{break}
        \ENDIF
        \STATE $\beta_n={r_{n+1}}^Tr_{n+1}/{r_n}^Tr_n$
        \STATE $p_{n+1}=r_{n+1}+\beta_np_n$
    \ENDWHILE
\end{algorithmic}
\end{algorithm}

\item \textbf{Function design}\\
Noticing that in this CG algorithm, we need to deal with some computations between constant, vectors and matrices. Thus, I design six methods below to help us solve these problems.
\begin{itemize}
\item{\emph{matvecDot}:} dot product of matrix with CSR format matrix and vector\\
Input 4 double type std::vector val, row\_ptr, col\_idx represents CSR format matrix, and vec. Output is a double type std::vector.
\item{\emph{vecAdd}:} add two vectors\\
Input 2 double type std::vector vec1, vec2. Output is a double type std::vector.
\item{\emph{vecSubtract}:} subtract two vectors\\
Input 2 double type std::vector vec1, vec2. Output is a double type std::vector.
\item{\emph{vecMul}:} multiply constant to vector\\
Input 1 double type number a, 1 double type std::vector vec. Output is a double type std::vector.
\item{\emph{vecDot}:} dot product of two vectors\\
Input 2 double type std::vector vec1, vec2. Output is a double type number.
\item{\emph{vecNorm}:} 2-norm of a vector\\
Input 1 double type std::vector vec. Output is a double type number.
\end{itemize}

\end{enumerate}
\end{document}
